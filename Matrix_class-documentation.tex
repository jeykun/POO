\documentclass[11pt]{article}
\usepackage[utf8]{inputenc}
\usepackage[T1]{fontenc}
\usepackage[left=2cm,right=2cm,top=2cm,bottom=2cm]{geometry}
%\usepackage[portuguese]{babel}
\usepackage{amsmath,amsfonts,amssymb}
\title{Documentation}
\author{Junior R. Ribeiro}
%\date{}
\usepackage[dvipsnames]{xcolor}
\usepackage{txfonts,classdiagram}
\usepackage[hidelinks]{hyperref}
\hypersetup{
colorlinks = true,
urlcolor = blue,
linkcolor = blue,
citecolor = blue,
}
%\setlength{\parskip}{12pt}
%\setlength{\parindent}{12pt}
%%%%%%%%%%%%%%%%%%%%%%%%%%%%%%%%%%%%%%%%%%%%%%%%%%%%%%%%%%%%%%%
% defs
\DeclareMathOperator{\trace}{\mathbf{tr}}
\DeclareMathOperator{\expm}{\mathbf{expm}}
\newcommand{\vb}[1]{\texttt{#1}}

%%%%%%%%%%%%%%%%%%%%%%%%%%%%%%%%%%%%%%%%%%%%%%%%%%%%%%%%%%%%%%%
\begin{document}

\maketitle
\listoftables

%\changeclasstext[Classe]
%\changelegendtext[Legenda]
%\changeclassendstext[termina aqui.]
%\changecontinuedfromprevious[continuação da página anterior]
%\changecontinuedonnext[continua...]
%\changedescriptcolor[colorname]













%ATTR1optional3mandtatory
%STUFF1optional3mandtatory
%METH1optional4mandtatory

%%%%%%%%%%%%%%%%%%%%%%%%%%%%% BEGIN WORKSPACE
\CLASS{\_\_workspace\_\_}
\STUFF[inline]{}{(empty)}{there is no global constants on workspace.}
\METHODS
\FUNC[]{+}{warn}{msg,endline*}{
	This function prints "WARN: msg" in red on terminal.

	>> msg is a string;
 
	>> endline is a boolean.
}
\FUNC{+}{info}{msg,endline*}{
	This function prints "INFO: msg" in blue on terminal.

	>> msg is a string;
 
	>> endline is a boolean.
}
\FUNC{+}{rcout}{msg,endline*}{
	This function prints "msg" in red on terminal.

	>> msg is a string;
 
	>> endline is a boolean.
}
\FUNC{+}{bcout}{msg,endline*}{
	This function prints "msg" in blue on terminal.

	>> msg is a string;
 
	>> endline is a boolean.
}
\LEGENDS
\STUFF[inline]{+}{public}{}
\STUFF[inline]{*}{optional}{}
\ENDCLASS\newpage
%%%%%%%%%%%%%%%%%%%%%%%%%%%%% END WORKSPACE







%%%%%%%%%%%%%%%%%%%%%%%%%%%%% BEGIN MATRIX
\CLASS{Matrix}
\ATTR[inline]{$-$}{me}{
	Pointer to pointer like **me (the Matrix itself).
}
\ATTR[inline]{$-$}{isdestroyed}{
	Boolean indicating whether the object was destroyed.
} 
\ATTR[inline]{$-$}{m}{
	The number of rows of the Matrix. A positive integer.
}
\ATTR[inline]{$-$}{n}{
	The number of columns of the Matrix. A positive integer.
}
\METHODS
\CONSTRUCTOR[inline]{+}{Matrix}{m,n}{
	The constructor method.
}
\METH{$+$}{throwisdestroyed}{functionName}{
This function raises an error and exits the program always when it is attempted to use a destroyed Matrix.

>> functionName is a string indicating the name of what function is attempting to use the Matrix.
}
\METH{$+$}{set}{i,j,value}{
...
}
\METH{$+$}{get}{i,j}{
...
}
\METH{$+$}{sum}{otherMatrix}{
...
}
\METH{$+$}{sub}{otherMatrix}{
...
}
\METH{$+$}{mul}{otherMatrix}{
...
}
\METH{$+$}{fromuser}{clearPrompt}{
...
}
\METH{$+$}{print}{ }{
...
}
\METH{$+$}{shape}{ }{
...
}
\METH{$+$}{shape1}{ }{
...
}
\METH{$+$}{shape2}{ }{
...
}
\METH{$+$}{destroy}{ }{
	This method desallocates the matrix and frees the memory. The integers m and n still remain on memory.
}
\LEGENDS
\STUFF[inline]{$+$}{public}{}
\STUFF[inline]{$-$}{private}{}
\ENDCLASS
%%%%%%%%%%%%%%%%%%%%%%%%%%%%% END MATRIX





%%%%%%%%%%%%%%%%%%%%%%%%%%
%\bibliography{refs.bib}{}%
%\bibliographystyle{plain}%
\end{document}





